\documentclass[12pt,a4paper]{article}
\usepackage[utf8]{inputenc}
\usepackage{amsmath}
\usepackage{amsfonts}
\usepackage{amssymb}
\usepackage[none]{hyphenat}
\usepackage{enumerate}
\usepackage[spanish]{babel}
\usepackage{graphicx}
\graphicspath{ {IMAGES/} }
\usepackage{float}
\usepackage{array}
\usepackage{colortbl}
\usepackage{enumitem}
\usepackage{pdfpages}
\usepackage{url, hyperref}
\renewcommand{\arraystretch}{1.8}
\pagestyle{headings}
\author{Juan Sebastian Gonzalez Camacho (1968220), Andrés Felipe Ruíz Buriticá (1968171), Jhoan Sebastian Rojas Holguin (1958337), Carlos Alberto Delgado Galeano (1968127), Jesus Alberto Gil Ayala (1968231)}
\title{Borvo MO - Requirements Specification}
\begin{document}
\begin{titlepage}
\centering
{\includegraphics[width=0.18 \textwidth]{logo.png} \par}
\vfill
{\bfseries\LARGE Universidad del Valle\par}
{\Large Sede Tuluá\par}
\vfill
{\scshape\Large Ingeniería de Sistemas \par}
\vfill
{\scshape\Huge Borvo - Medicinae Operam \par}
\vfill
{\itshape\Large Avance 2 - Proyecto Final Desarrollo de Software I \par}
\vfill
{\Large Autores: \par}
{\Large Andrés Felipe Ruíz Buriticá - 1968171 \par}
{\Large Carlos Alberto Delgado Galeano - 1968127 \par}
{\Large Jesús Alberto Gil Ayala - 1968231 \par}
{\Large Jhoan Sebastian Rojas Holguin - 1958337 \par}
{\Large Juan Sebastian González Camacho - 1968220 \par}
\vfill
{\Large 5 de Diciembre del 2022 \par}
\end{titlepage}
\tableofcontents
\newpage
\section{Artefactos Ágiles}
\subsection{Roles Equipo Scrum}
\begin{center}
\begin{tabular}{|m{5cm}|m{9cm}|}
\hline
\textbf{Rol} & Product Owner. \\
\hline
\textbf{Integrante Asignado} & Juan Sebastian Gonzalez Camacho. \\
\hline
\textbf{Descripción del Rol} & El Product Owner es el encargado de optimizar y maximizar el valor del producto, siendo la persona encargada de gestionar el flujo de valor del producto a través del Product Backlog. \\
\hline
\textbf{Funciones del Rol} & \begin{enumerate}[noitemsep]
								 \item Definir los objetivos del producto.
								 \item Determinar las características del producto.
								 \item Crear el Backlog.
								 \item Crear Historias de Usuario.
								 \item Priorizar y gestionar el Backlog.
								 \item Supervizar las etapas de desarrollo del producto.	
							\end{enumerate} \\
\hline
\end{tabular}
\vspace{5mm}

\begin{tabular}{|m{5cm}|m{9cm}|}
\hline
\textbf{Rol} & Scrum Master. \\
\hline
\textbf{Integrante Asignado} & Andres Felipe Ruiz Buritica. \\
\hline
\textbf{Descripción del Rol} & El Scrum Master se encarga de gestionar el proceso Scrum y ayudar a eliminar impedimentos que puedan afectar a la entrega del producto. Además, se encarga de las labores de mentoring y formación, coaching y de facilitar reuniones y eventos si es necesario. \\
\hline
\textbf{Funciones del Rol} & \begin{enumerate}[noitemsep]
								 \item Gestionar el proceso.
								 \item Eliminar impedimentos.
								 \item Planear reuniones.
								 \item Dirigir al equipo de desarrollo.	
							\end{enumerate} \\
\hline
\end{tabular}
\vspace{5mm}

\begin{tabular}{|m{5cm}|m{9cm}|}
\hline
\textbf{Rol} & Equipo de Desarrollo. \\
\hline
\textbf{Integrante Asignado} & Carlos Alberto Delgado Galeano, Jesus Alberto Gil Ayala, Jhoan Sebastian Rojas Holguin. \\
\hline
\textbf{Descripción del Rol} & El equipo de desarrollo suele estar formado por entre 3 a 9 profesionales que se encargan de desarrollar el producto, auto-organizándose y auto-gestionándose para conseguir entregar un incremento de software al final del ciclo de desarrollo. \\
\hline
\textbf{Funciones del Rol} & \begin{enumerate}[noitemsep]
								 \item Desarrollar el producto.
								 \item Revisar el Backlog y los Sprint Plan.
								 \item Entregar las tareas terminadas.
								 \item Comunicarse constantemente con el Scrum Master.	
							\end{enumerate} \\
\hline
\end{tabular}
\vspace{5mm}


\end{center}
\newpage
\subsection{Sprint 0}
\begin{enumerate}
\item Levantamiento de requerimientos funcionales y no funcionales.
\item Elaboración del diagrama de casos de uso y especificación de casos de uso.
\item Diseño del modelo entidad relación y modelo relacional.
\item Diseño del diagrama de clases.
\item Asignación de roles Scrum.
\item Elaboración del User Story Mapping.
\item Redacción de Historias de Usuario.
\item Diseño del Backlog.
\item Elaboración del Release Plan.
\item Elaboración del Sprint Plan.
\item Cronograma de ceremonias ágiles.
\item Elección de herramientas, gestores, frameworks y demás elementos a utilizar.
\end{enumerate}
\subsection{Backlog}
\url{https://mementocoding.atlassian.net/jira/software/projects/BMO/boards/1/backlog}
\subsection{Historias de Usuario}
\begin{center}
\begin{tabular}{|>{\columncolor[RGB]{215, 215, 215}} p{10cm} >{\columncolor[RGB]{215, 215, 215}} c >{\columncolor[RGB]{215, 215, 215}} p{2.5cm}|}
\hline 
\textbf{Historia de Usuario \#1}

Gestionar afiliados & & \textbf{{\Large EPIC}} \\ 
\textbf{Descripción}

Como administrador, quiero gestionar la información de los afiliados, para poder mantener los datos actualizados en el sistema. &  & \textbf{Prioridad}

Alta\\

\textbf{Criterio de Aceptación}

\begin{itemize}
\item Dado que una afiliado no existe en el sistema, cuando el administrador lo registra, entonces el sistema muestra un mensaje indicando que la operación ha sido exitosa.
\item Dado que un afiliado tiene beneficiarios, cuando el administrador los registra, entonces el sistema muestra un mensaje indicando que la operación ha sido un éxito.
\item Dado que un afiliado tiene distintos datos, cuando el administrador ve los detalles, entonces el sistema muestra una página con toda la información del cotizante, incluyendo sus beneficiarios.
\item Dado que un afiliado tiene información desactualizada, cuando el administrador la edita, entonces el sistema muestra un formulario de edición y al final un mensaje indicando que los datos han sido modificados.
\item Dado que un afiliado ya no tiene relación con la EPS, cuando el administrador lo elimina, el sistema desactiva la cuenta y muestra un mensaje de éxito.
\end{itemize} & & \textbf{Estimación}

13 \\ 

\textbf{Definición de Hecho}

\begin{itemize}
\item Código producido para esta funcionalidad.
\item Esta funcionalidad se implementa sin errores.
\item Se cumplen todos los criterios de aceptación.
\item Función aprobada por el Product Owner.
\item Pruebas escritas y aprobadas.
\item La funcionalidad está desplegada en producción.
\end{itemize} & & \\
\hline
\end{tabular}
\vspace{5mm}

\begin{tabular}{| p{10cm} c p{2.5cm}|}
\hline 
\textbf{Historia de Usuario \#2}

Ver información de un afiliado & & \textbf{{\Large STORY}} \\ 
\textbf{Descripción}

Como administrador, quiero ver los datos de un cotizante y sus beneficiarios, para poder saber qué información contiene el sistema sobre ellos. &  & \textbf{Prioridad}

Alta\\

\textbf{Criterio de Aceptación}

\begin{itemize}
\item Dado que existen distintos cotizantes, cuando el administrador accede a los detalles de alguno, el sistema le muestra una página con toda la información relacionada del cotizante y sus beneficiarios.
\end{itemize} & & \textbf{Estimación}

5 \\ 

\textbf{Definición de Hecho}

\begin{itemize}
\item Código producido para esta funcionalidad.
\item Esta funcionalidad se implementa sin errores.
\item Se cumplen todos los criterios de aceptación.
\item Función aprobada por el Product Owner.
\item Pruebas escritas y aprobadas.
\item La funcionalidad está desplegada en producción.
\end{itemize} & & \\
\hline 
\end{tabular}
\vspace{5mm}

\begin{tabular}{| p{10cm} c p{2.5cm}|}
\hline 
\textbf{Historia de Usuario \#3}

Añadir afiliados & & \textbf{{\Large STORY}} \\ 
\textbf{Descripción}

Como administrador, quiero añadir un cotizante y sus beneficiarios, para
poder gestionar su información posteriormente. &  & \textbf{Prioridad}

Alta\\

\textbf{Criterio de Aceptación}

\begin{itemize}
\item Dado que el afiliado no existe en el sistema, cuando el
administrador lo añade, entonces el sistema le muestra una
notificación indicando que la operación ha sido exitosa.
\item Dado que el afiliado ya existe en el sistema,
cuando el administrador lo añade, entonces el sistema le
muestra una notificación indicando que ya existe.
\end{itemize} & & \textbf{Estimación}

5 \\ 

\textbf{Definición de Hecho}

\begin{itemize}
\item Código producido para esta funcionalidad.
\item Esta funcionalidad se implementa sin errores.
\item Se cumplen todos los criterios de aceptación.
\item Función aprobada por el Product Owner.
\item Pruebas escritas y aprobadas.
\item La funcionalidad está desplegada en producción.
\end{itemize} & & \\
\hline  
\end{tabular}
\vspace{5mm}

\begin{tabular}{| p{10cm} c p{2.5cm}|}
\hline 
\textbf{Historia de Usuario \#4}

Editar afiliados & & \textbf{{\Large STORY}} \\ 
\textbf{Descripción}

Como administrador, quiero editar la información de un cotizante y sus
beneficiarios, para poder mantener actualizado el registro de ellos en el
sistema. &  & \textbf{Prioridad}

Alta\\

\textbf{Criterio de Aceptación}

\begin{itemize}
\item Dado que un afiliado tiene información desactualizada, cuando el
administrador edita sus datos, entonces el sistema le muestra
una notificación indicando que el registro se ha actualizado.
\end{itemize} & & \textbf{Estimación}

5 \\ 

\textbf{Definición de Hecho}

\begin{itemize}
\item Código producido para esta funcionalidad.
\item Esta funcionalidad se implementa sin errores.
\item Se cumplen todos los criterios de aceptación.
\item Función aprobada por el Product Owner.
\item Pruebas escritas y aprobadas.
\item La funcionalidad está desplegada en producción.
\end{itemize} & & \\
\hline  
\end{tabular}
\vspace{5mm}

\begin{tabular}{| p{10cm} c p{2.5cm}|}
\hline 
\textbf{Historia de Usuario \#5}

Eliminar afiliados & & \textbf{{\Large STORY}} \\ 
\textbf{Descripción}

Como administrador, quiero eliminar un afiliado, para poder mantener el registro del
sistema actualizado. &  & \textbf{Prioridad}

Alta\\

\textbf{Criterio de Aceptación}

\begin{itemize}
\item Dado que un afiliado está mal registrado, cuando el
administrador la elimina, entonces el sistema muestra una notificación señalando que la
operación ha sido exitosa.
\end{itemize} & & \textbf{Estimación}

5 \\ 

\textbf{Definición de Hecho}

\begin{itemize}
\item Código producido para esta funcionalidad.
\item Esta funcionalidad se implementa sin errores.
\item Se cumplen todos los criterios de aceptación.
\item Función aprobada por el Product Owner.
\item Pruebas escritas y aprobadas.
\item La funcionalidad está desplegada en producción.
\end{itemize} & & \\
\hline  
\end{tabular}
\vspace{5mm}

\begin{tabular}{|>{\columncolor[RGB]{215, 215, 215}} p{10cm} >{\columncolor[RGB]{215, 215, 215}} c >{\columncolor[RGB]{215, 215, 215}} p{2.5cm}|}
\hline 
\textbf{Historia de Usuario \#6}

Gestionar IPS & & \textbf{{\Large EPIC}} \\ 
\textbf{Descripción}

Como administrador, quiero gestionar los contratos de las IPS, para poder
mantener el registro del sistema actualizado. &  & \textbf{Prioridad}

Alta\\

\textbf{Criterio de Aceptación}

\begin{itemize}
\item Dado que una IPS no existe en el sistema, cuando el
administrador la añade, entonces el sistema muestra un mensaje
indicando que la operación ha sido exitosa.
\item Dado que hay muchas IPS en el sistema, cuando el administrador
intenta ver los detalles de alguna, el sistema le muestra toda la
información guardada relacionada.
\item Dado que una IPS ya no tiene contrato con la EPS, cuando el
administrador la elimina, entonces el sistema muestra un
mensaje indicando que la operación ha sido exitosa.
\item Dado que los datos de una IPS han cambiado, cuando el
administrador intenta editarla, el sistema le muestra un
formulación de edición y al terminar, un mensaje indicando que
la operación ha sido exitosa.
\end{itemize} & & \textbf{Estimación}

8 \\ 

\textbf{Definición de Hecho}

\begin{itemize}
\item Código producido para esta funcionalidad.
\item Esta funcionalidad se implementa sin errores.
\item Se cumplen todos los criterios de aceptación.
\item Función aprobada por el Product Owner.
\item Pruebas escritas y aprobadas.
\item La funcionalidad está desplegada en producción.
\end{itemize} & & \\
\hline 
\end{tabular}
\vspace{5mm}

\begin{tabular}{| p{10cm} c p{2.5cm}|}
\hline 
\textbf{Historia de Usuario \#7}

Ver información de una IPS & & \textbf{{\Large STORY}} \\ 
\textbf{Descripción}

Como administrador, quiero ver los datos de una IPS, para poder saber
qué información contiene el sistema sobre ella. &  & \textbf{Prioridad}

Alta\\

\textbf{Criterio de Aceptación}

\begin{itemize}
\item Dado que existen distintas IPS, cuando el administrador accede a
los detalles de alguna, el sistema le muestra una página con toda
la información relacionada.
\end{itemize} & & \textbf{Estimación}

3 \\ 

\textbf{Definición de Hecho}

\begin{itemize}
\item Código producido para esta funcionalidad.
\item Esta funcionalidad se implementa sin errores.
\item Se cumplen todos los criterios de aceptación.
\item Función aprobada por el Product Owner.
\item Pruebas escritas y aprobadas.
\item La funcionalidad está desplegada en producción.
\end{itemize} & & \\
\hline  
\end{tabular}
\vspace{5mm}

\begin{tabular}{| p{10cm} c p{2.5cm}|}
\hline 
\textbf{Historia de Usuario \#8}

Añadir IPS & & \textbf{{\Large STORY}} \\ 
\textbf{Descripción}

Como administrador, quiero añadir un contrato con una IPS, para poder
gestionar las órdenes de servicio que ella ofrece. &  & \textbf{Prioridad}

Alta\\

\textbf{Criterio de Aceptación}

\begin{itemize}
\item Dado que la IPS no existe en el sistema, cuando el administrador
la añade, entonces el sistema le muestra una notificación
indicando que la operación ha sido exitosa.
\item Dado que la IPS ya existe en el sistema, cuando el administrador
la añade, entonces el sistema le muestra una notificación
indicando que ya existe otra IPS con el mismo NIT y razón social.
\end{itemize} & & \textbf{Estimación}

3 \\ 

\textbf{Definición de Hecho}

\begin{itemize}
\item Código producido para esta funcionalidad.
\item Esta funcionalidad se implementa sin errores.
\item Se cumplen todos los criterios de aceptación.
\item Función aprobada por el Product Owner.
\item Pruebas escritas y aprobadas.
\item La funcionalidad está desplegada en producción.
\end{itemize} & & \\
\hline  
\end{tabular}
\vspace{5mm}

\begin{tabular}{| p{10cm} c p{2.5cm}|}
\hline 
\textbf{Historia de Usuario \#9}

Editar IPS & & \textbf{{\Large STORY}} \\ 
\textbf{Descripción}

Como administrador, quiero editar la información de una IPS, para poder
mantener actualizado el registro de ella en el sistema. &  & \textbf{Prioridad}

Alta\\

\textbf{Criterio de Aceptación}

\begin{itemize}
\item Dado que una IPS tiene información desactualizada, cuando el
administrador edita sus datos, entonces el sistema le muestra
una notificación indicando que el registro se ha actualizado.
\end{itemize} & & \textbf{Estimación}

3 \\ 

\textbf{Definición de Hecho}

\begin{itemize}
\item Código producido para esta funcionalidad.
\item Esta funcionalidad se implementa sin errores.
\item Se cumplen todos los criterios de aceptación.
\item Función aprobada por el Product Owner.
\item Pruebas escritas y aprobadas.
\item La funcionalidad está desplegada en producción.
\end{itemize} & & \\
\hline 
\end{tabular}
\vspace{5mm}

\begin{tabular}{| p{10cm} c p{2.5cm}|}
\hline 
\textbf{Historia de Usuario \#10}

Eliminar IPS & & \textbf{{\Large STORY}} \\ 
\textbf{Descripción}

Como administrador, quiero eliminar una IPS que ya no tiene contrato,
para poder mantener el registro del sistema actualizado. &  & \textbf{Prioridad}

Alta\\

\textbf{Criterio de Aceptación}

\begin{itemize}
\item Dado que una IPS ya no tiene relación con la EPS, cuando el
administrador la elimina, entonces el sistema muestra una
notificación señalando que la operación ha sido exitosa.
\end{itemize} & & \textbf{Estimación}

3 \\ 

\textbf{Definición de Hecho}

\begin{itemize}
\item Código producido para esta funcionalidad.
\item Esta funcionalidad se implementa sin errores.
\item Se cumplen todos los criterios de aceptación.
\item Función aprobada por el Product Owner.
\item Pruebas escritas y aprobadas.
\item La funcionalidad está desplegada en producción.
\end{itemize} & & \\
\hline  
\end{tabular}
\vspace{5mm}

\begin{tabular}{|>{\columncolor[RGB]{215, 215, 215}} p{10cm} >{\columncolor[RGB]{215, 215, 215}} c >{\columncolor[RGB]{215, 215, 215}} p{2.5cm}|}
\hline 
\textbf{Historia de Usuario \#11}

Gestionar empresa & & \textbf{{\Large EPIC}} \\ 
\textbf{Descripción}


Como administrador, quiero gestionar la información de las empresas,
para poder mantener el registro del sistema actualizado. &  & \textbf{Prioridad}

Alta\\

\textbf{Criterio de Aceptación}

\begin{itemize}
\item Dado que una empresa no existe en el sistema, cuando el
administrador la añade, entonces el sistema muestra un mensaje
indicando que la operación ha sido exitosa.
\item Dado que hay muchas empresas en el sistema, cuando el
administrador intenta ver los detalles de alguna, el sistema le
muestra toda la información guardada relacionada.
\item Dado que una empresa ya no tiene relación con el EPS, cuando el
administrador la elimina, entonces el sistema muestra un
mensaje indicando que la operación ha sido exitosa.
\item Dado que la información de una empresa ha cambiado, cuando
el administrador intenta editarla, el sistema le muestra un
formulación de edición y al terminar, un mensaje indicando que
la operación ha sido exitosa.
\end{itemize} & & \textbf{Estimación}

8 \\ 

\textbf{Definición de Hecho}

\begin{itemize}
\item Código producido para esta funcionalidad.
\item Esta funcionalidad se implementa sin errores.
\item Se cumplen todos los criterios de aceptación.
\item Función aprobada por el Product Owner.
\item Pruebas escritas y aprobadas.
\item La funcionalidad está desplegada en producción.
\end{itemize} & & \\
\hline 
\end{tabular}
\vspace{5mm}

\begin{tabular}{| p{10cm} c p{2.5cm}|}
\hline 
\textbf{Historia de Usuario \#12}

Ver información de una empresa & & \textbf{{\Large STORY}} \\ 
\textbf{Descripción}

Como administrador, quiero ver la información de una empresa, para
poder saber qué información contiene el sistema sobre ella. &  & \textbf{Prioridad}

Alta\\

\textbf{Criterio de Aceptación}

\begin{itemize}
\item Dado que existen distintas empresas, cuando el administrador
accede a los detalles de alguna, el sistema le muestra una página
con toda la información relacionada.
\end{itemize} & & \textbf{Estimación}

3 \\ 

\textbf{Definición de Hecho}

\begin{itemize}
\item Código producido para esta funcionalidad.
\item Esta funcionalidad se implementa sin errores.
\item Se cumplen todos los criterios de aceptación.
\item Función aprobada por el Product Owner.
\item Pruebas escritas y aprobadas.
\item La funcionalidad está desplegada en producción.
\end{itemize} & & \\
\hline 
\end{tabular}
\vspace{5mm}

\begin{tabular}{| p{10cm} c p{2.5cm}|}
\hline 
\textbf{Historia de Usuario \#13}

Añadir empresa & & \textbf{{\Large STORY}} \\ 
\textbf{Descripción}

Como administrador, quiero añadir nuevas empresas al sistema, para
poder gestionar los afiliados que pertenezcan a ella. &  & \textbf{Prioridad}

Alta\\

\textbf{Criterio de Aceptación}

\begin{itemize}
\item Dado que la empresa no existe en el sistema, cuando el
administrador la añade, entonces el sistema le muestra una
notificación indicando que la operación ha sido exitosa.
\item Dado que la empresa ya existe en el sistema, cuando el
administrador la añade, entonces el sistema le muestra una
notificación indicando que ya existe otra empresa con el mismo
NIT y razón social.
\end{itemize} & & \textbf{Estimación}

3 \\ 

\textbf{Definición de Hecho}

\begin{itemize}
\item Código producido para esta funcionalidad.
\item Esta funcionalidad se implementa sin errores.
\item Se cumplen todos los criterios de aceptación.
\item Función aprobada por el Product Owner.
\item Pruebas escritas y aprobadas.
\item La funcionalidad está desplegada en producción.
\end{itemize} & & \\
\hline  
\end{tabular}
\vspace{5mm}

\begin{tabular}{| p{10cm} c p{2.5cm}|}
\hline 
\textbf{Historia de Usuario \#14}

Editar empresa & & \textbf{{\Large STORY}} \\ 
\textbf{Descripción}

Como administrador, quiero editar la información de una empresa, para
poder mantener actualizado el registro de empresas del sistema. &  & \textbf{Prioridad}

Alta\\

\textbf{Criterio de Aceptación}

\begin{itemize}
\item Dado que una empresa tiene información desactualizada, cuando
el administrador edita sus datos, entonces el sistema le muestra
una notificación indicando que el registro se ha actualizado.
\end{itemize} & & \textbf{Estimación}

3 \\ 

\textbf{Definición de Hecho}

\begin{itemize}
\item Código producido para esta funcionalidad.
\item Esta funcionalidad se implementa sin errores.
\item Se cumplen todos los criterios de aceptación.
\item Función aprobada por el Product Owner.
\item Pruebas escritas y aprobadas.
\item La funcionalidad está desplegada en producción.
\end{itemize} & & \\
\hline  
\end{tabular}
\vspace{5mm}

\begin{tabular}{| p{10cm} c p{2.5cm}|}
\hline 
\textbf{Historia de Usuario \#15}

Eliminar empresa & & \textbf{{\Large STORY}} \\ 
\textbf{Descripción}

Como administrador, quiero eliminar una empresa del sistema, para
poder mantener el registro del sistema actualizado. &  & \textbf{Prioridad}

Alta\\

\textbf{Criterio de Aceptación}

\begin{itemize}
\item Dado que una empresa ya no tiene relación con la EPS, cuando el
administrador la elimina, entonces el sistema muestra una
notificación señalando que la operación ha sido exitosa.
\end{itemize} & & \textbf{Estimación}

3 \\ 

\textbf{Definición de Hecho}

\begin{itemize}
\item Código producido para esta funcionalidad.
\item Esta funcionalidad se implementa sin errores.
\item Se cumplen todos los criterios de aceptación.
\item Función aprobada por el Product Owner.
\item Pruebas escritas y aprobadas.
\item La funcionalidad está desplegada en producción.
\end{itemize} & & \\
\hline  
\end{tabular}
\vspace{5mm}

\begin{tabular}{|>{\columncolor[RGB]{215, 215, 215}} p{10cm} >{\columncolor[RGB]{215, 215, 215}} c >{\columncolor[RGB]{215, 215, 215}} p{2.5cm}|}
\hline 
\textbf{Historia de Usuario \#16}

Reportar novedad & & \textbf{{\Large EPIC}} \\ 
\textbf{Descripción}

Como banco registrado en el sistema, quiero reportar la vinculación o
retiro de un empleado, para poder activar o desactivar la cuenta del
cotizante dentro del sistema. &  & \textbf{Prioridad}

Alta\\

\textbf{Criterio de Aceptación}

\begin{itemize}
\item Dado que el empleado se retiró de la empresa, cuando el banco
reporta su retiro, entonces el sistema cambia el estado del
cotizante a retirado e inactiva la cuenta de usuario asociada.
\item Dado que el empleado se vinculó a la empresa, cuando el banco
reporta su vinculación, entonces el sistema crea su cuenta.
\end{itemize} & & \textbf{Estimación}

5 \\ 

\textbf{Definición de Hecho}

\begin{itemize}
\item Código producido para esta funcionalidad.
\item Esta funcionalidad se implementa sin errores.
\item Se cumplen todos los criterios de aceptación.
\item Función aprobada por el Product Owner.
\item Pruebas escritas y aprobadas.
\item La funcionalidad está desplegada en producción.
\end{itemize} & & \\
\hline  
\end{tabular}
\vspace{5mm}

\begin{tabular}{| p{10cm} c p{2.5cm}|}
\hline 
\textbf{Historia de Usuario \#17}

Reportar vinculación de cotizante & & \textbf{{\Large STORY}} \\ 
\textbf{Descripción}

Como banco registrado en el sistema, quiero reportar la vinculación de
un empleado, para poder activar la cuenta del cotizante dentro del
sistema. &  & \textbf{Prioridad}

Alta\\

\textbf{Criterio de Aceptación}

\begin{itemize}
\item Dado que el empleado se vinculó a la empresa, cuando el banco
reporta su vinculación, entonces el sistema crea su cuenta.
\end{itemize} & & \textbf{Estimación}

3 \\ 

\textbf{Definición de Hecho}

\begin{itemize}
\item Código producido para esta funcionalidad.
\item Esta funcionalidad se implementa sin errores.
\item Se cumplen todos los criterios de aceptación.
\item Función aprobada por el Product Owner.
\item Pruebas escritas y aprobadas.
\item La funcionalidad está desplegada en producción.
\end{itemize} & & \\
\hline  
\end{tabular}
\vspace{5mm}

\begin{tabular}{| p{10cm} c p{2.5cm}|}
\hline 
\textbf{Historia de Usuario \#18}

Reportar retiro de cotizante & & \textbf{{\Large STORY}} \\ 
\textbf{Descripción}

Como banco registrado en el sistema, quiero reportar el retiro de un
empleado, para poder desactivar la cuenta del cotizante dentro del
sistema. &  & \textbf{Prioridad}

Alta\\

\textbf{Criterio de Aceptación}

\begin{itemize}
\item Dado que el empleado se retiró de la empresa, cuando el banco
reporta su retiro, entonces el sistema cambia el estado del
cotizante a retirado e inactiva la cuenta de usuario asociada.
\end{itemize} & & \textbf{Estimación}

3 \\ 

\textbf{Definición de Hecho}

\begin{itemize}
\item Código producido para esta funcionalidad.
\item Esta funcionalidad se implementa sin errores.
\item Se cumplen todos los criterios de aceptación.
\item Función aprobada por el Product Owner.
\item Pruebas escritas y aprobadas.
\item La funcionalidad está desplegada en producción.
\end{itemize} & & \\
\hline  
\end{tabular}
\vspace{5mm}

\begin{tabular}{|>{\columncolor[RGB]{215, 215, 215}} p{10cm} >{\columncolor[RGB]{215, 215, 215}} c >{\columncolor[RGB]{215, 215, 215}} p{2.5cm}|}
\hline 
\textbf{Historia de Usuario \#19}

Generar reportes & & \textbf{{\Large EPIC}} \\ 
\textbf{Descripción}

Como administrador, quiero generar distintos reportes, para poder ver la
información actual que contiene el sistema. &  & \textbf{Prioridad}

Media\\

\textbf{Criterio de Aceptación}

\begin{itemize}
\item Dado que hay afiliados con distinto estado, cuando el
administrador genera el reporte, entonces el sistema le muestra
una lista de todos los afiliados con el estado actual de ellos.
\item Dado que hay órdenes de servicio por pacientes, cuando el
administrador genera el reporte, entonces el sistema le muestra
una lista de todos los afiliados con las órdenes de servicios que
han sacado hasta la fecha.
\item Dado que hay cotizantes por empresa, cuando el administrador
genera el reporte, entonces el sistema muestra una lista de todos
los afiliados junto con la información de la empresa a la que
perteneces.
\end{itemize} & & \textbf{Estimación}

2 \\ 

\textbf{Definición de Hecho}

\begin{itemize}
\item Código producido para esta funcionalidad.
\item Esta funcionalidad se implementa sin errores.
\item Se cumplen todos los criterios de aceptación.
\item Función aprobada por el Product Owner.
\item Pruebas escritas y aprobadas.
\item La funcionalidad está desplegada en producción.
\end{itemize} & & \\
\hline 
\end{tabular}
\vspace{5mm}

\begin{tabular}{| p{10cm} c p{2.5cm}|}
\hline 
\textbf{Historia de Usuario \#20}

Generar reporte de afiliados por estado & & \textbf{{\Large STORY}} \\ 
\textbf{Descripción}

Como administrador, quiero generar un reporte de afiliados por estado,
para poder ver la información actual que contiene el sistema. &  & \textbf{Prioridad}

Media\\

\textbf{Criterio de Aceptación}

\begin{itemize}
\item Dado que hay afiliados con distinto estado, cuando el
administrador genera el reporte, entonces el sistema le muestra
una lista de todos los afiliados con el estado actual de ellos.
\end{itemize} & & \textbf{Estimación}

2 \\ 

\textbf{Definición de Hecho}

\begin{itemize}
\item Código producido para esta funcionalidad.
\item Esta funcionalidad se implementa sin errores.
\item Se cumplen todos los criterios de aceptación.
\item Función aprobada por el Product Owner.
\item Pruebas escritas y aprobadas.
\item La funcionalidad está desplegada en producción.
\end{itemize} & & \\
\hline 
\end{tabular}
\vspace{5mm}

\begin{tabular}{| p{10cm} c p{2.5cm}|}
\hline 
\textbf{Historia de Usuario \#21}

Generar reporte de órdenes de servicio por paciente & & \textbf{{\Large STORY}} \\ 
\textbf{Descripción}

Como administrador, quiero generar reportes de órdenes de servicio por
paciente, para poder ver la información actual que contiene el sistema. &  & \textbf{Prioridad}

Media\\

\textbf{Criterio de Aceptación}

\begin{itemize}
\item Dado que hay órdenes de servicio por pacientes, cuando el
administrador genera el reporte, entonces el sistema le muestra
una lista de todos los afiliados con las órdenes de servicios que
han sacado hasta la fecha.
\end{itemize} & & \textbf{Estimación}

2 \\ 

\textbf{Definición de Hecho}

\begin{itemize}
\item Código producido para esta funcionalidad.
\item Esta funcionalidad se implementa sin errores.
\item Se cumplen todos los criterios de aceptación.
\item Función aprobada por el Product Owner.
\item Pruebas escritas y aprobadas.
\item La funcionalidad está desplegada en producción.
\end{itemize} & & \\
\hline  
\end{tabular}
\vspace{5mm}

\begin{tabular}{| p{10cm} c p{2.5cm}|}
\hline 
\textbf{Historia de Usuario \#22}

Generar reportes de cotizantes por empresa & & \textbf{{\Large STORY}} \\ 
\textbf{Descripción}

Como administrador, quiero generar reportes de cotizantes por empresa,
para poder ver la información actual que contiene el sistema. &  & \textbf{Prioridad}

Media\\

\textbf{Criterio de Aceptación}

\begin{itemize}
\item Dado que hay cotizantes por empresa, cuando el administrador
genera el reporte, entonces el sistema muestra una lista de todos
los afiliados junto con la información de la empresa a la que
perteneces.
\end{itemize} & & \textbf{Estimación}

2 \\ 

\textbf{Definición de Hecho}

\begin{itemize}
\item Código producido para esta funcionalidad.
\item Esta funcionalidad se implementa sin errores.
\item Se cumplen todos los criterios de aceptación.
\item Función aprobada por el Product Owner.
\item Pruebas escritas y aprobadas.
\item La funcionalidad está desplegada en producción.
\end{itemize} & & \\
\hline  
\end{tabular}
\vspace{5mm}

\begin{tabular}{|>{\columncolor[RGB]{215, 215, 215}} p{10cm} >{\columncolor[RGB]{215, 215, 215}} c >{\columncolor[RGB]{215, 215, 215}} p{2.5cm}|}
\hline 
\textbf{Historia de Usuario \#23}

Reportar pago de aportes & & \textbf{{\Large EPIC}} \\ 
\textbf{Descripción}

Como banco registrado en el sistema, quiero reportar el pago de
cotizantes, para poder mantener la información financiera actualizada
cada mes. &  & \textbf{Prioridad}

Alta\\

\textbf{Criterio de Aceptación}

\begin{itemize}
\item Dado que hay pagos de muchos cotizantes por reportar, cuando
el banco solicita hacer el reporte, entonces el sistema le muestra
un formulario por cada cotizante.
\item Dado que hay pagos de un cotizante por reportar, cuando el
banco solicita hacer el reporte, entonces el sistema le muestra un
formulario para el cotizante.
\item Dado que el banco está llenando la información del reporte,
cuando ingresa mal algún dato, entonces el sistema le notifica el
error para corregirlo.
\end{itemize} & & \textbf{Estimación}

5 \\ 

\textbf{Definición de Hecho}

\begin{itemize}
\item Código producido para esta funcionalidad.
\item Esta funcionalidad se implementa sin errores.
\item Se cumplen todos los criterios de aceptación.
\item Función aprobada por el Product Owner.
\item Pruebas escritas y aprobadas.
\item La funcionalidad está desplegada en producción.
\end{itemize} & & \\
\hline  
\end{tabular}
\vspace{5mm}

\begin{tabular}{| p{10cm} c p{2.5cm}|}
\hline 
\textbf{Historia de Usuario \#24}

Reportar pago de aportes individual & & \textbf{{\Large STORY}} \\ 
\textbf{Descripción}

Como banco registrado en el sistema, quiero reportar el pago de aportes
de cotizantes de forma individual, para poder registrar uno por uno
manualmente cuando sean muy pocos. &  & \textbf{Prioridad}

Alta\\

\textbf{Criterio de Aceptación}

\begin{itemize}
\item Dado que hay empresas que hacen pocos pagos de aportes, cuando el banco solicita
hacer el reporte, entonces el sistema le permite registrarlos de forma individual
mediante un formulario.
\end{itemize} & & \textbf{Estimación}

3 \\ 

\textbf{Definición de Hecho}

\begin{itemize}
\item Código producido para esta funcionalidad.
\item Esta funcionalidad se implementa sin errores.
\item Se cumplen todos los criterios de aceptación.
\item Función aprobada por el Product Owner.
\item Pruebas escritas y aprobadas.
\item La funcionalidad está desplegada en producción.
\end{itemize} & & \\
\hline  
\end{tabular}
\vspace{5mm}

\begin{tabular}{| p{10cm} c p{2.5cm}|}
\hline 
\textbf{Historia de Usuario \#25}

Reportar pago de aportes en bloque & & \textbf{{\Large STORY}} \\ 
\textbf{Descripción}

Como banco registrado en el sistema, quiero reportar los pagos de aportes
de los cotizantes de una empresa en bloque, para poder registrar con mayor
agilidad cuando sean reportados muchos pagos. &  & \textbf{Prioridad}

Media\\

\textbf{Criterio de Aceptación}

\begin{itemize}
\item Dado que hay empresas que reportan muchos pagos, cuando el banco solicita
hacer el reporte, el sistema le permite cargar desde un archivo varios reportes.
\end{itemize} & & \textbf{Estimación}

3 \\ 

\textbf{Definición de Hecho}

\begin{itemize}
\item Código producido para esta funcionalidad.
\item Esta funcionalidad se implementa sin errores.
\item Se cumplen todos los criterios de aceptación.
\item Función aprobada por el Product Owner.
\item Pruebas escritas y aprobadas.
\item La funcionalidad está desplegada en producción.
\end{itemize} & & \\
\hline 
\end{tabular}
\vspace{5mm}

\begin{tabular}{|>{\columncolor[RGB]{215, 215, 215}} p{10cm} >{\columncolor[RGB]{215, 215, 215}} c >{\columncolor[RGB]{215, 215, 215}} p{2.5cm}|}
\hline 
\textbf{Historia de Usuario \#26}

Gestionar orden de servicio & & \textbf{{\Large EPIC}} \\ 
\textbf{Descripción}


Como administrador, quiero gestionar la información de las órdenes de servicio,
para poder mantener el registro del sistema actualizado. &  & \textbf{Prioridad}

Alta\\

\textbf{Criterio de Aceptación}

\begin{itemize}
\item Dado que una orden de servicio no existe en el sistema, cuando el
administrador la añade, entonces el sistema muestra un mensaje
indicando que la operación ha sido exitosa.
\item Dado que hay muchas órdenes de servicio en el sistema, cuando el
administrador intenta ver los detalles de alguna, el sistema le
muestra toda la información guardada relacionada.
\item Dado que una órden de servicio ya no tiene relación con el EPS, cuando el
administrador la elimina, entonces el sistema muestra un
mensaje indicando que la operación ha sido exitosa.
\item Dado que la información de una órden de servicio ha cambiado, cuando
el administrador intenta editarla, el sistema le muestra un
formulación de edición y al terminar, un mensaje indicando que
la operación ha sido exitosa.
\end{itemize} & & \textbf{Estimación}

8 \\ 

\textbf{Definición de Hecho}

\begin{itemize}
\item Código producido para esta funcionalidad.
\item Esta funcionalidad se implementa sin errores.
\item Se cumplen todos los criterios de aceptación.
\item Función aprobada por el Product Owner.
\item Pruebas escritas y aprobadas.
\item La funcionalidad está desplegada en producción.
\end{itemize} & & \\
\hline  
\end{tabular}
\vspace{5mm}

\begin{tabular}{| p{10cm} c p{2.5cm}|}
\hline 
\textbf{Historia de Usuario \#27}

Ver información de una orden de servicio & & \textbf{{\Large STORY}} \\ 
\textbf{Descripción}

Como administrador, quiero ver la información de una orden de servicio, para
poder saber qué información contiene el sistema sobre ella. &  & \textbf{Prioridad}

Alta\\

\textbf{Criterio de Aceptación}

\begin{itemize}
\item Dado que existen distintas órdenes de servicio, cuando el administrador
accede a los detalles de alguna, el sistema le muestra una página
con toda la información relacionada.
\end{itemize} & & \textbf{Estimación}

3 \\ 

\textbf{Definición de Hecho}

\begin{itemize}
\item Código producido para esta funcionalidad.
\item Esta funcionalidad se implementa sin errores.
\item Se cumplen todos los criterios de aceptación.
\item Función aprobada por el Product Owner.
\item Pruebas escritas y aprobadas.
\item La funcionalidad está desplegada en producción.
\end{itemize} & & \\
\hline  
\end{tabular}
\vspace{5mm}

\begin{tabular}{| p{10cm} c p{2.5cm}|}
\hline 
\textbf{Historia de Usuario \#28}

Añadir orden de servicio & & \textbf{{\Large STORY}} \\ 
\textbf{Descripción}

Como administrador, quiero añadir nuevas órdenes de servicio al sistema, para
poder gestionar los afiliados que las tengan asignadas. &  & \textbf{Prioridad}

Alta\\

\textbf{Criterio de Aceptación}

\begin{itemize}
\item Dado que la orden de servicio no existe en el sistema, cuando el
administrador la añade, entonces el sistema le muestra una
notificación indicando que la operación ha sido exitosa.
\item Dado que la orden de servicio ya existe en el sistema, cuando el
administrador la añade, entonces el sistema le muestra una
notificación indicando que ya existe otra orden de servicio con el mismo
código.
\end{itemize} & & \textbf{Estimación}

3 \\ 

\textbf{Definición de Hecho}

\begin{itemize}
\item Código producido para esta funcionalidad.
\item Esta funcionalidad se implementa sin errores.
\item Se cumplen todos los criterios de aceptación.
\item Función aprobada por el Product Owner.
\item Pruebas escritas y aprobadas.
\item La funcionalidad está desplegada en producción.
\end{itemize} & & \\
\hline  
\end{tabular}
\vspace{5mm}

\begin{tabular}{| p{10cm} c p{2.5cm}|}
\hline 
\textbf{Historia de Usuario \#29}

Editar orden de servicio & & \textbf{{\Large STORY}} \\ 
\textbf{Descripción}

Como administrador, quiero editar la información de una orden de servicio, para
poder mantener actualizado el registro de órdenes de servicio del sistema. &  & \textbf{Prioridad}

Alta\\

\textbf{Criterio de Aceptación}

\begin{itemize}
\item Dado que una orden de servicio tiene información desactualizada, cuando
el administrador edita sus datos, entonces el sistema le muestra
una notificación indicando que el registro se ha actualizado.
\end{itemize} & & \textbf{Estimación}

3 \\ 

\textbf{Definición de Hecho}

\begin{itemize}
\item Código producido para esta funcionalidad.
\item Esta funcionalidad se implementa sin errores.
\item Se cumplen todos los criterios de aceptación.
\item Función aprobada por el Product Owner.
\item Pruebas escritas y aprobadas.
\item La funcionalidad está desplegada en producción.
\end{itemize} & & \\
\hline  
\end{tabular}
\vspace{5mm}

\begin{tabular}{| p{10cm} c p{2.5cm}|}
\hline 
\textbf{Historia de Usuario \#30}

Eliminar orden de servicio & & \textbf{{\Large STORY}} \\ 
\textbf{Descripción}

Como administrador, quiero eliminar una orden de servicio del sistema, para
poder mantener el registro del sistema actualizado. &  & \textbf{Prioridad}

Alta\\

\textbf{Criterio de Aceptación}

\begin{itemize}
\item Dado que una orden de servicio ya no tiene relación con la EPS, cuando el
administrador la elimina, entonces el sistema muestra una
notificación señalando que la operación ha sido exitosa.
\end{itemize} & & \textbf{Estimación}

3 \\ 

\textbf{Definición de Hecho}

\begin{itemize}
\item Código producido para esta funcionalidad.
\item Esta funcionalidad se implementa sin errores.
\item Se cumplen todos los criterios de aceptación.
\item Función aprobada por el Product Owner.
\item Pruebas escritas y aprobadas.
\item La funcionalidad está desplegada en producción.
\end{itemize} & & \\
\hline  
\end{tabular}
\vspace{5mm}

\begin{tabular}{| p{10cm} c p{2.5cm}|}
\hline 
\textbf{Historia de Usuario \#31}

Consultar perfil & & \textbf{{\Large STORY}} \\ 
\textbf{Descripción}

Como cotizante, quiero ver los detalles de mi cuenta, para poder saber
qué información mía está guardada en el sistema. &  & \textbf{Prioridad}

Alta\\

\textbf{Criterio de Aceptación}

\begin{itemize}
\item Dado que el cotizante está actualmente autenticado, cuando el
usuario pide ver los detalles de la cuenta, entonces el sistema le
muestra una página con toda la información del cotizante.
\item Dado que el cotizante está actualmente autenticado y tiene
beneficiarios registrados, cuando el usuario pide ver los detalles
de la cuenta, entonces el sistema le muestra una página con toda
la información del cotizante y de sus beneficiarios.
\end{itemize} & & \textbf{Estimación}

3 \\ 

\textbf{Definición de Hecho}

\begin{itemize}
\item Código producido para esta funcionalidad.
\item Esta funcionalidad se implementa sin errores.
\item Se cumplen todos los criterios de aceptación.
\item Función aprobada por el Product Owner.
\item Pruebas escritas y aprobadas.
\item La funcionalidad está desplegada en producción.
\end{itemize} & & \\
\hline 
\end{tabular}
\vspace{5mm}

\begin{tabular}{| p{10cm} c p{2.5cm}|}
\hline 
\textbf{Historia de Usuario \#32}

Iniciar sesión & & \textbf{{\Large STORY}} \\ 
\textbf{Descripción}

Como usuario existente, quiero iniciar sesión con mis credenciales, para
poder acceder al sistema. &  & \textbf{Prioridad}

Alta\\

\textbf{Criterio de Aceptación}

\begin{itemize}
\item Dado que las credenciales son correctas, cuando el usuario
intenta iniciar sesión, entonces el sistema lo lleva a la página de
inicio.
\item Dado que las credenciales son incorrectas, cuando el usuario
ntenta iniciar sesión, entonces el sistema muestra una
notificación indicando el error.
\end{itemize} & & \textbf{Estimación}

3 \\ 

\textbf{Definición de Hecho}

\begin{itemize}
\item Código producido para esta funcionalidad.
\item Esta funcionalidad se implementa sin errores.
\item Se cumplen todos los criterios de aceptación.
\item Función aprobada por el Product Owner.
\item Pruebas escritas y aprobadas.
\item La funcionalidad está desplegada en producción.
\end{itemize} & & \\
\hline  
\end{tabular}
\vspace{5mm}

\end{center}
\subsection{Seguimiento de Jira}
\url{https://mementocoding.atlassian.net/jira/software/projects/BMO/boards/1}
\subsection{Ceremonias Ágiles}
\begin{enumerate}
\item \textbf{Sprint Planning:} Durante esta reunión, el product owner presenta el Product Backlog actualizado que el equipo de desarrollo se encarga de estimar, además de intentar clarificar aquellos ítems que crea necesarios. Cada miembro del equipo de desarrollo selecciona los ítems que va a trabajar durante el sprint. Esta reunión tiene una duración de 2 a 3 horas.
\item \textbf{Daily Scrum:} Es una reunión diaria de 15 minutos en la que participa exclusivamente el equipo de desarrollo. Se comparte una retroalimentación breve entre todos y se resuelven dudas si es necesario.
\item \textbf{Sprint Review:} Es la reunión que ocurre al final del Sprint, generalmente el último domingo del Sprint, donde el product owner y el equipo de desarrollo presentan el incremento terminado para su inspección y adaptación correspondientes. En esta reunión organizada por el product owner se estudia cuál es la situación y se actualiza el Product Backlog con las nuevas condiciones que puedan afectar al proceso.
\item \textbf{Sprint Retrospective:} La retrospectiva ocurre justo después del Sprint Review. Se realiza conjuntamente con el Sprint Planning, siendo la retrospectiva la parte inicial de la reunión. El objetivo de la retrospectiva es hacer de reflexión sobre el último Sprint e identificar posibles mejoras para el próximo. 
\end{enumerate}
\subsection{Plan de Pruebas}
\begin{center}
\begin{tabular}{|m{5cm}|m{9cm}|}
\hline
\multicolumn{2}{|c|}{\textbf{CP-01}} \\
\hline
\textbf{Historia de Usuario} & Iniciar Sesión. \\
\hline
\textbf{Nombre del Caso} & Iniciar Sesión. \\
\hline
\textbf{Objetivo de la Prueba} & Verificar si se puede iniciar sesión de forma correcta. \\
\hline
\textbf{Pre-Condiciones} & El Cotizante debe estar en el panel cotizante. \\
\hline
\textbf{Pasos} & Paso 1: El usuario debe ingresar a la opción cotizante.

Paso 2: Ingresar las credenciales en la pagina principal de borvo.

Paso 3: Presionar enter e ingresar.
 \\
\hline
\textbf{Resultado Esperado} & El usuario ha hecho login exitosamente. \\
\hline
\textbf{Resultado Obtenido} & El usuario ha hecho login exitosamente. \\
\hline
\textbf{Sprint} & 1. \\
\hline
\textbf{Estado del Caso} & Aprobado. \\
\hline
\end{tabular}
\vspace{5mm}

\begin{tabular}{|m{5cm}|m{9cm}|}
\hline
\multicolumn{2}{|c|}{\textbf{CP-02}} \\
\hline
\textbf{Historia de Usuario} & Ver información de un afiliado. \\
\hline
\textbf{Nombre del Caso} & Ver información de un afiliado. \\
\hline
\textbf{Objetivo de la Prueba} & Verificar si el sistema muestra todos los datos de un afiliado.\\
\hline
\textbf{Pre-Condiciones} & El usuario-Administrador debió haber ingresado al sistema mediante el panel admin. \\
\hline
\textbf{Pasos} & Paso 1: El usuario-administrador debe presionar el botón de consulta que aparece en el sistema en la fila del usuario al que va a consultar.
 \\
\hline
\textbf{Resultado Esperado} & Ver los datos del afiliado en una ventana. \\
\hline
\textbf{Resultado Obtenido} & * \\
\hline
\textbf{Sprint} & 1. \\
\hline
\textbf{Estado del Caso} & Pendiente. \\
\hline
\end{tabular}
\vspace{5mm}

\begin{tabular}{|m{5cm}|m{9cm}|}
\hline
\multicolumn{2}{|c|}{\textbf{CP-03}} \\
\hline
\textbf{Historia de Usuario} & Añadir afiliados. \\
\hline
\textbf{Nombre del Caso} & Añadir afiliados. \\
\hline
\textbf{Objetivo de la Prueba} & Verificar si se puede añadir de manera exitosa un afiliado nuevo a la base de datos de la EPS. \\
\hline
\textbf{Pre-Condiciones} & El Administrador debe estar en el panel gestionar afiliados. \\
\hline
\textbf{Pasos} & Paso 1: Dar click en el botón añadir afiliado.

Paso 2: Llenar el formulario con los datos del afiliado.

Paso 3: Dar click en guardar.\\
\hline
\textbf{Resultado Esperado} & El afiliado fue añadido de manera exitosa a la base de datos de la EPS. \\
\hline
\textbf{Resultado Obtenido} & * \\
\hline
\textbf{Sprint} & 1. \\
\hline
\textbf{Estado del Caso} & Pendiente. \\
\hline
\end{tabular}
\vspace{5mm}

\begin{tabular}{|m{5cm}|m{9cm}|}
\hline
\multicolumn{2}{|c|}{\textbf{CP-04}} \\
\hline
\textbf{Historia de Usuario} & Editar afiliado. \\
\hline
\textbf{Nombre del Caso} & Editar afiliado. \\
\hline
\textbf{Objetivo de la Prueba} & Verificar que se pueden editar los diferentes datos de los afiliados que ya se encuentran registrados en la base de datos para mantenerlos actualizados. \\
\hline
\textbf{Pre-Condiciones} & El Administrador debe estar en el panel gestionar afiliados. \\
\hline
\textbf{Pasos} & Paso 1: Dar click en el botón editar que se encuentra en la fila del afiliado que se desea editar.

Paso 2: Editar los campos en el formulario de los datos que se desean actualizar.

Paso 3: Dar click en guardar.
 \\
\hline
\textbf{Resultado Esperado} & Mostrar el afiliado en la tabla con los datos actualizados. \\
\hline
\textbf{Resultado Obtenido} & * \\
\hline
\textbf{Sprint} & 1. \\
\hline
\textbf{Estado del Caso} & Pendiente. \\
\hline
\end{tabular}
\vspace{5mm}

\begin{tabular}{|m{5cm}|m{9cm}|}
\hline
\multicolumn{2}{|c|}{\textbf{CP-05}} \\
\hline
\textbf{Historia de Usuario} & Eliminar afiliado. \\
\hline
\textbf{Nombre del Caso} & Eliminar afiliado. \\
\hline
\textbf{Objetivo de la Prueba} & Verificar que se elimina un afiliado de la base de datos. \\
\hline
\textbf{Pre-Condiciones} & El Administrador debe estar en el panel gestionar afiliados. \\
\hline
\textbf{Pasos} & Paso 1: Dar click en el botón eliminar que se encuentra en la fila del afiliado que se desea eliminar. \\
\hline
\textbf{Resultado Esperado} & El afiliado ya no aparece en la base de datos. \\
\hline
\textbf{Resultado Obtenido} & * \\
\hline
\textbf{Sprint} & 1. \\
\hline
\textbf{Estado del Caso} & Pendiente. \\
\hline
\end{tabular}
\vspace{5mm}

\begin{tabular}{|m{5cm}|m{9cm}|}
\hline
\multicolumn{2}{|c|}{\textbf{CP-06}} \\
\hline
\textbf{Historia de Usuario} & Ver información IPS. \\
\hline
\textbf{Nombre del Caso} & Ver información IPS. \\
\hline
\textbf{Objetivo de la Prueba} & Verificar que se puede visualizar toda la información de las IPS registradas en la base de datos. \\
\hline
\textbf{Pre-Condiciones} & El Administrador debe estar en el panel gestionar IPS. \\
\hline
\textbf{Pasos} & Paso 1: Dar click en el botón de visualizar que se encuentra en la fila de la IPS de la cual quiere ver toda la información. \\
\hline
\textbf{Resultado Esperado} &  Ver los datos de la IPS en una ventana.\\
\hline
\textbf{Resultado Obtenido} & * \\
\hline
\textbf{Sprint} & 1. \\
\hline
\textbf{Estado del Caso} & Pendiente. \\
\hline
\end{tabular}
\vspace{5mm}

\begin{tabular}{|m{5cm}|m{9cm}|}
\hline
\multicolumn{2}{|c|}{\textbf{CP-07}} \\
\hline
\textbf{Historia de Usuario} & Añadir IPS. \\
\hline
\textbf{Nombre del Caso} & Añadir IPS. \\
\hline
\textbf{Objetivo de la Prueba} & Verificar que se puede añadir una IPS de forma correcta. \\
\hline
\textbf{Pre-Condiciones} & El Administrador debe estar en el panel gestionar IPS. \\
\hline
\textbf{Pasos} & Paso 1: Dar click en el botón añadir IPS.

Paso 2: Llenar el formulario con los datos de la IPS.

Paso 3: Dar click en guardar. \\
\hline
\textbf{Resultado Esperado} & La IPS fue agregada de manera exitosa. \\
\hline
\textbf{Resultado Obtenido} & * \\
\hline
\textbf{Sprint} & 1. \\
\hline
\textbf{Estado del Caso} & Pendiente. \\
\hline
\end{tabular}
\vspace{5mm}

\begin{tabular}{|m{5cm}|m{9cm}|}
\hline
\multicolumn{2}{|c|}{\textbf{CP-08}} \\
\hline
\textbf{Historia de Usuario} & Editar IPS. \\
\hline
\textbf{Nombre del Caso} & Editar IPS. \\
\hline
\textbf{Objetivo de la Prueba} & Verificar que se puede editar la información de las IPS. \\
\hline
\textbf{Pre-Condiciones} & El Administrador debe estar en el panel gestionar IPS. \\
\hline
\textbf{Pasos} & Paso 1: Dar click en el botón editar que se encuentra en la fila de la IPS de la cual se desea editar información.

Paso 2: Llenar los campos en el formulario de los datos que se desean actualizar.

Paso 3: Dar click en guardar.\\
\hline
\textbf{Resultado Esperado} & Mostrar la IPS en la tabla con los datos actualizados. \\
\hline
\textbf{Resultado Obtenido} & * \\
\hline
\textbf{Sprint} & 1. \\
\hline
\textbf{Estado del Caso} & Pendiente. \\
\hline
\end{tabular}
\vspace{5mm}

\begin{tabular}{|m{5cm}|m{9cm}|}
\hline
\multicolumn{2}{|c|}{\textbf{CP-09}} \\
\hline
\textbf{Historia de Usuario} & Eliminar IPS. \\
\hline
\textbf{Nombre del Caso} & Eliminar IPS. \\
\hline
\textbf{Objetivo de la Prueba} &  Verificar que se elimina una IPS de la base de datos.\\
\hline
\textbf{Pre-Condiciones} & El Administrador debe estar en el panel gestionar IPS. \\
\hline
\textbf{Pasos} & Paso 1: Dar click en el botón eliminar que se encuentra en la fila de la IPS que se desea eliminar. \\
\hline
\textbf{Resultado Esperado} & La IPS ya no aparece en la base de datos. \\
\hline
\textbf{Resultado Obtenido} & * \\
\hline
\textbf{Sprint} & 1. \\
\hline
\textbf{Estado del Caso} & Pendiente. \\
\hline
\end{tabular}
\vspace{5mm}

\begin{tabular}{|m{5cm}|m{9cm}|}
\hline
\multicolumn{2}{|c|}{\textbf{CP-10}} \\
\hline
\textbf{Historia de Usuario} & Ver información de empresa. \\
\hline
\textbf{Nombre del Caso} & Ver información de empresa. \\
\hline
\textbf{Objetivo de la Prueba} & Verificar que se puede visualizar la información de una empresa de forma correcta. \\
\hline
\textbf{Pre-Condiciones} & El Administrador debe estar en el panel gestionar empresa. \\
\hline
\textbf{Pasos} & * \\
\hline
\textbf{Resultado Esperado} & Ver una tabla con la información de todas las empresas. \\
\hline
\textbf{Resultado Obtenido} & Ver una tabla con la información de todas las empresas. \\
\hline
\textbf{Sprint} & 1. \\
\hline
\textbf{Estado del Caso} & Aprobado. \\
\hline
\end{tabular}
\vspace{5mm}

\begin{tabular}{|m{5cm}|m{9cm}|}
\hline
\multicolumn{2}{|c|}{\textbf{CP-11}} \\
\hline
\textbf{Historia de Usuario} & Añadir empresa. \\
\hline
\textbf{Nombre del Caso} & Añadir empresa. \\
\hline
\textbf{Objetivo de la Prueba} & Verificar que se puede añadir una empresa de forma correcta. \\
\hline
\textbf{Pre-Condiciones} &  El Administrador debe estar en el panel gestionar empresa.\\
\hline
\textbf{Pasos} & Paso 1: Dar click en el botón añadir empresa.

Paso 2: Llenar el formulario con los datos de la empresa.

Paso 3: Dar click en guardar. \\
\hline
\textbf{Resultado Esperado} & La empresa fue añadida de manera exitosa. \\
\hline
\textbf{Resultado Obtenido} & La empresa fue añadida de manera exitosa. \\
\hline
\textbf{Sprint} & 1. \\
\hline
\textbf{Estado del Caso} & Aprobado. \\
\hline
\end{tabular}
\vspace{5mm}

\begin{tabular}{|m{5cm}|m{9cm}|}
\hline
\multicolumn{2}{|c|}{\textbf{CP-12}} \\
\hline
\textbf{Historia de Usuario} & Editar empresa. \\
\hline
\textbf{Nombre del Caso} & Editar empresa. \\
\hline
\textbf{Objetivo de la Prueba} & Verificar que se puede editar la información de las empresas. \\
\hline
\textbf{Pre-Condiciones} & El Administrador debe estar en el panel gestionar empresa. \\
\hline
\textbf{Pasos} & Paso 1: Dar click en el botón editar que se encuentra en la fila de la empresa de la cual se desea editar información.

Paso 2: Llenar los campos en el formulario de los datos que se desean actualizar.

Paso 3: Dar click en guardar.\\
\hline
\textbf{Resultado Esperado} & Mostrar la empresa en la tabla con los datos actualizados. \\
\hline
\textbf{Resultado Obtenido} & Se muestra la empreas con los datos actualizados. \\
\hline
\textbf{Sprint} & 1. \\
\hline
\textbf{Estado del Caso} & Aprobado. \\
\hline
\end{tabular}
\vspace{5mm}

\begin{tabular}{|m{5cm}|m{9cm}|}
\hline
\multicolumn{2}{|c|}{\textbf{CP-13}} \\
\hline
\textbf{Historia de Usuario} & Eliminar empresa. \\
\hline
\textbf{Nombre del Caso} & Eliminar empresa. \\
\hline
\textbf{Objetivo de la Prueba} &  Verificar que se elimina una empresa de la base de datos.\\
\hline
\textbf{Pre-Condiciones} & El Administrador debe estar en el panel gestionar empresa. \\
\hline
\textbf{Pasos} & Paso 1: Dar click en el botón eliminar que se encuentra en la fila de la empresa que se desea eliminar. \\
\hline
\textbf{Resultado Esperado} & La empresa ya no aparece en la base de datos. \\
\hline
\textbf{Resultado Obtenido} & La empresa ya no se muestra en la base de datos. \\
\hline
\textbf{Sprint} & 1. \\
\hline
\textbf{Estado del Caso} & Aprobado. \\
\hline
\end{tabular}
\vspace{5mm}

\begin{tabular}{|m{5cm}|m{9cm}|}
\hline
\multicolumn{2}{|c|}{\textbf{CP-14}} \\
\hline
\textbf{Historia de Usuario} & Ver información de orden de servicio. \\
\hline
\textbf{Nombre del Caso} & Ver información de orden de servicio. \\
\hline
\textbf{Objetivo de la Prueba} & Verificar que se puede visualizar la información de una orden de servicio de forma correcta. \\
\hline
\textbf{Pre-Condiciones} & El Administrador debe estar en el panel gestionar ordenes de servicio. \\
\hline
\textbf{Pasos} & Paso 1: Dar click en el botón de ver información que se encuentra en la fila de la órden de servicio que se desea consultar. \\
\hline
\textbf{Resultado Esperado} & Ver una ventana con la información de la orden de servicio. \\
\hline
\textbf{Resultado Obtenido} & * \\
\hline
\textbf{Sprint} & 1. \\
\hline
\textbf{Estado del Caso} & Pendiente. \\
\hline
\end{tabular}
\vspace{5mm}

\begin{tabular}{|m{5cm}|m{9cm}|}
\hline
\multicolumn{2}{|c|}{\textbf{CP-15}} \\
\hline
\textbf{Historia de Usuario} & Añadir orden de servicio. \\
\hline
\textbf{Nombre del Caso} & Añadir orden de servicio. \\
\hline
\textbf{Objetivo de la Prueba} & Verificar que se puede añadir una orden de servicio de forma correcta. \\
\hline
\textbf{Pre-Condiciones} &  El Administrador debe estar en el panel gestionar orden de servicio.\\
\hline
\textbf{Pasos} & Paso 1: Dar click en el botón añadir orden de servicio.

Paso 2: Llenar el formulario con los datos de la orden de servicio.

Paso 3: Dar click en guardar. \\
\hline
\textbf{Resultado Esperado} & La orden de servicio fue añadida de manera exitosa. \\
\hline
\textbf{Resultado Obtenido} & * \\
\hline
\textbf{Sprint} & 1. \\
\hline
\textbf{Estado del Caso} & Pendiente. \\
\hline
\end{tabular}
\vspace{5mm}

\begin{tabular}{|m{5cm}|m{9cm}|}
\hline
\multicolumn{2}{|c|}{\textbf{CP-16}} \\
\hline
\textbf{Historia de Usuario} & Editar orden de servicio. \\
\hline
\textbf{Nombre del Caso} & Editar orden de servicio. \\
\hline
\textbf{Objetivo de la Prueba} & Verificar que se puede editar la información de las ordenes de servicio. \\
\hline
\textbf{Pre-Condiciones} & El Administrador debe estar en el panel gestionar ordenes de servicio. \\
\hline
\textbf{Pasos} & Paso 1: Dar click en el botón editar que se encuentra en la fila de la orden de servicio de la cual se desea editar información.

Paso 2: Llenar los campos en el formulario de los datos que se desean actualizar.

Paso 3: Dar click en guardar.\\
\hline
\textbf{Resultado Esperado} & Mostrar la orden de servicio en la tabla con los datos actualizados. \\
\hline
\textbf{Resultado Obtenido} & * \\
\hline
\textbf{Sprint} & 1. \\
\hline
\textbf{Estado del Caso} & Pendiente. \\
\hline
\end{tabular}
\vspace{5mm}

\begin{tabular}{|m{5cm}|m{9cm}|}
\hline
\multicolumn{2}{|c|}{\textbf{CP-17}} \\
\hline
\textbf{Historia de Usuario} & Eliminar orden de servicio. \\
\hline
\textbf{Nombre del Caso} & Eliminar orden de servicio. \\
\hline
\textbf{Objetivo de la Prueba} &  Verificar que se elimina una orden de servicio de la base de datos.\\
\hline
\textbf{Pre-Condiciones} & El Administrador debe estar en el panel gestionar ordenes de servicio. \\
\hline
\textbf{Pasos} & Paso 1: Dar click en el botón eliminar que se encuentra en la fila de la orden de servicio que se desea eliminar. \\
\hline
\textbf{Resultado Esperado} & La orden de servicio ya no aparece en la base de datos. \\
\hline
\textbf{Resultado Obtenido} & *. \\
\hline
\textbf{Sprint} & 1. \\
\hline
\textbf{Estado del Caso} & Pendiente. \\
\hline
\end{tabular}
\vspace{5mm}

\begin{tabular}{|m{5cm}|m{9cm}|}
\hline
\multicolumn{2}{|c|}{\textbf{CP-18}} \\
\hline
\textbf{Historia de Usuario} & Consultar perfil. \\
\hline
\textbf{Nombre del Caso} & Consultar perfil. \\
\hline
\textbf{Objetivo de la Prueba} & Verificar que se puede visualizar toda la información de un cotizante.\\
\hline
\textbf{Pre-Condiciones} & El cotizante debe estar registrado y hacer login de manera exitosa.\\
\hline
\textbf{Pasos} & * \\
\hline
\textbf{Resultado Esperado} & Encontrar toda la información de un cotizante. \\
\hline
\textbf{Resultado Obtenido} & Se encontró toda la información relacionada con el cotizante y sus beneficiarios. \\
\hline
\textbf{Sprint} & 1. \\
\hline
\textbf{Estado del Caso} & Aprobado. \\
\hline
\end{tabular}
\vspace{5mm}

\begin{tabular}{|m{5cm}|m{9cm}|}
\hline
\multicolumn{2}{|c|}{\textbf{CP-19}} \\
\hline
\textbf{Historia de Usuario} & Reportar retiro cotizante \\
\hline
\textbf{Nombre del Caso} & Reportar retiro cotizante \\
\hline
\textbf{Objetivo de la Prueba} &  Reportar el retiro de un empleado para desactivar la cuenta del cotizante\\
\hline
\textbf{Pre-Condiciones} & El banco debe estar en el panel de banco\\
\hline
\textbf{Pasos} & Paso 1: Dar click en el boton reportar novedad para desplegar los tipos de reportes implementados.

Paso 2: Dar click sobre el boton Reportar Retiro.

Paso 3: Ingresa el DNI del cotizante al que se le quiere reportar un retiro.

Paso 4: Click en guardar.  \\
\hline
\textbf{Resultado Esperado} & El estado del cotizante cambia a inactivo cuando se retira de la empresa. \\
\hline
\textbf{Resultado Obtenido} & * \\
\hline
\textbf{Sprint} & 1. \\
\hline
\textbf{Estado del Caso} & Pendiente.\\
\hline
\end{tabular}
\vspace{5mm}

\begin{tabular}{|m{5cm}|m{9cm}|}
\hline
\multicolumn{2}{|c|}{\textbf{CP-20}} \\
\hline
\textbf{Historia de Usuario} & Reportar vinculación del cotizante \\
\hline
\textbf{Nombre del Caso} & Reportar vinculación del cotizante \\
\hline
\textbf{Objetivo de la Prueba} & Reportar la vinculación de un empleado para poder activar la cuenta del cotizante dentro del sistema \\
\hline
\textbf{Pre-Condiciones} & El banco debe estar en el panel reportar novedad \\
\hline
\textbf{Pasos} & Paso 1:Click en el panel reportar novedad donde se desplegaran los dos tipos de reportes implementados.

Paso 2: Click en el boton reportar Vinculación.

Paso 3: El banco ingresa el DNI del cotizante y el NIT de la empresa a la que pertenece el cotizante.

Paso 4: Click en guardar.
\\
\hline
\textbf{Resultado Esperado} & Se vincula el cotizante con la empresa de forma correcta y su estado pasa a activo. \\
\hline
\textbf{Resultado Obtenido} & * \\
\hline
\textbf{Sprint} & 1. \\
\hline
\textbf{Estado del Caso} & Pendiente. \\
\hline
\end{tabular}
\vspace{5mm}

\begin{tabular}{|m{5cm}|m{9cm}|}
\hline
\multicolumn{2}{|c|}{\textbf{CP-21}} \\
\hline
\textbf{Historia de Usuario} & Reportar pago de aportes en bloque.\\
\hline
\textbf{Nombre del Caso} & Reportar pago de aportes en bloque.\\
\hline
\textbf{Objetivo de la Prueba} & Verificar que el pago de aportes en bloque se hace de manera satisfactoria.\\
\hline
\textbf{Pre-Condiciones} & El banco debe de estar en el panel reportar pago.\\
\hline
\textbf{Pasos} & Paso 1: Ir al botón de reportar pago de aportes en bloque.

Paso 2: Se importa el csv en el botón importar datos.

Paso 3: Se le envía el aviso en la aplicación al banco que ya se reporto el pago de aportes en bloque.\\
\hline
\textbf{Resultado Esperado} & El banco recibe el aviso de reporte de pago de aportes satisfactoriamente.\\
\hline
\textbf{Resultado Obtenido} & *\\
\hline
\textbf{Sprint} & 1. \\
\hline
\textbf{Estado del Caso} & Pendiente .\\
\hline
\end{tabular}
\vspace{5mm}


\begin{tabular}{|m{5cm}|m{9cm}|}
\hline
\multicolumn{2}{|c|}{\textbf{CP-22}} \\
\hline
\textbf{Historia de Usuario} & Reportar pago de aportes individuales.\\
\hline
\textbf{Nombre del Caso} & Reportar pago de aportes individuales.\\
\hline
\textbf{Objetivo de la Prueba} & Verificar que el pago de aportes individuales se hace de manera satisfactoria.\\
\hline
\textbf{Pre-Condiciones} & El banco debe de estar en el panel reportar pago.\\
\hline
\textbf{Pasos} & Paso 1: Ir al botón de reportar pago de aportes individuales.

Paso 2: Se rellenan los datos de pago del cotizante.

Paso 3: Se le envía el aviso en la aplicación al banco que ya se reporto el pago de aportes individuales.\\
\hline
\textbf{Resultado Esperado} & El banco recibe el aviso de reporte de pago de aportes satisfactoriamente.\\
\hline
\textbf{Resultado Obtenido} & *\\
\hline
\textbf{Sprint} & 1. \\
\hline
\textbf{Estado del Caso} & Pendiente. \\
\hline
\end{tabular}
\vspace{5mm}

\begin{tabular}{|m{5cm}|m{9cm}|}
\hline
\multicolumn{2}{|c|}{\textbf{CP-23}} \\
\hline
\textbf{Historia de Usuario} & Generar reporte de afiliados por estado.\\
\hline
\textbf{Nombre del Caso} & Generar reporte de afiliados por estado.\\
\hline
\textbf{Objetivo de la Prueba} & Verificar que se genera el reporte de afiliados por estado correctamente.\\
\hline
\textbf{Pre-Condiciones} & El administrador esté en el panel de generar reportes. \\
\hline
\textbf{Pasos} & Paso 1: Clic botón de generar reportes de afiliados por estado.

Paso 2: seleccionar el estado que se quiere filtrar (Activo, inactivo o retirado).

Paso 3: Clic en Generar. \\
\hline
\textbf{Resultado Esperado} & Se genera un reporte de todos los afiliados según el estado seleccionado.\\
\hline
\textbf{Resultado Obtenido} & *\\
\hline
\textbf{Sprint} & 1. \\
\hline
\textbf{Estado del Caso} & Pendiente.\\
\hline
\end{tabular}
\vspace{5mm}

\begin{tabular}{|m{5cm}|m{9cm}|}
\hline
\multicolumn{2}{|c|}{\textbf{CP-24}} \\
\hline
\textbf{Historia de Usuario} & Generar reporte de órdenes de servicio por paciente.\\
\hline
\textbf{Nombre del Caso} & Generar reporte de órdenes de servicio por paciente. \\
\hline
\textbf{Objetivo de la Prueba} & Generar órdenes de servicio por paciente para poder ver la información actual que contiene el sistema. \\
\hline
\textbf{Pre-Condiciones} & El administrador debe estar en el panel Genera Reportes.\\
\hline
\textbf{Pasos} & Paso 1:Clic en el panel General Reportes donde se desplegaran los  tipos de reportes implementados.

Paso 2: Clic en el botón órdenes por paciente.

Paso 3: El administrador ingresa el DNI del paciente.

Paso 4: Generar.
\\
\hline
\textbf{Resultado Esperado} & Se muestran todas las ordenes del paciente.\\
\hline
\textbf{Resultado Obtenido} & * \\
\hline
\textbf{Sprint} & 1 \\
\hline
\textbf{Estado del Caso} & Pendiente \\
\hline
\end{tabular}
\vspace{5mm}

\begin{tabular}{|m{5cm}|m{9cm}|}
\hline
\multicolumn{2}{|c|}{\textbf{CP-25}} \\
\hline
\textbf{Historia de Usuario} & Generar reportes de cotizantes por empresa. \\
\hline
\textbf{Nombre del Caso} & Generar reportes de cotizantes por empresa.\\
\hline
\textbf{Objetivo de la Prueba} & Verificar que se genera el reporte de los cotizantes por empresa completamente.  \\
\hline
\textbf{Pre-Condiciones} & El administrador debe estar en el panel de generar reportes. \\
\hline
\textbf{Pasos} & Paso 1:Click en el panel de generar reportes.

Paso 2: Click botón de generar reportes de cotizantes por empresa.

Paso 3: El administrador ingresa el NIT de la empresa.

Paso 4: Click en generar.
\\
\hline
\textbf{Resultado Esperado} & Se genera una lista de cotizantes por empresa.\\
\hline
\textbf{Resultado Obtenido} & * \\
\hline
\textbf{Sprint} & 1. \\
\hline
\textbf{Estado del Caso} & Pendiente. \\
\hline
\end{tabular}
\vspace{5mm}

\end{center}
\end{document}